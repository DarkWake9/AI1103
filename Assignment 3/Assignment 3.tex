\documentclass[journal,12pt,twocolumn]{IEEEtran}

\usepackage{setspace}
\usepackage{gensymb}
\singlespacinzg
\usepackage[cmex10]{amsmath}

\usepackage{amsthm}

\usepackage{mathrsfs}
\usepackage{txfonts}
\usepackage{stfloats}
\usepackage{bm}
\usepackage{cite}
\usepackage{cases}
\usepackage{subfig}

\usepackage{longtable}
\usepackage{multirow}

\usepackage{enumitem}
\usepackage{mathtools}
\usepackage{steinmetz}
\usepackage{tikz}
\usepackage{circuitikz}
\usepackage{verbatim}
\usepackage{tfrupee}
\usepackage[breaklinks=true]{hyperref}
\usepackage{graphicx}
\usepackage{tkz-euclide}

\usetikzlibrary{calc,math}
\usepackage{listings}
    \usepackage{color}                                            %%
    \usepackage{array}                                            %%
    \usepackage{longtable}                                        %%
    \usepackage{calc}                                             %%
    \usepackage{multirow}                                         %%
    \usepackage{hhline}                                           %%
    \usepackage{ifthen}                                           %%
    \usepackage{lscape}     
\usepackage{multicol}
\usepackage{chngcntr}

\DeclareMathOperator*{\Res}{Res}

\renewcommand\thesection{\arabic{section}}
\renewcommand\thesubsection{\thesection.\arabic{subsection}}
\renewcommand\thesubsubsection{\thesubsection.\arabic{subsubsection}}

\renewcommand\thesectiondis{\arabic{section}}
\renewcommand\thesubsectiondis{\thesectiondis.\arabic{subsection}}
\renewcommand\thesubsubsectiondis{\thesubsectiondis.\arabic{subsubsection}}


\hyphenation{op-tical net-works semi-conduc-tor}
\def\inputGnumericTable{}                                 %%

\lstset{
%language=C,
frame=single, 
breaklines=true,
columns=fullflexible
}
\begin{document}
\title{Assignment 3}
\author{Vibhavasu Pasumarti - EP20BTECH11015}
\maketitle
\newpage
\bigskip
\renewcommand{\thefigure}{\theenumi}
\renewcommand{\thetable}{\theenumi}
Download all python codes from 
\begin{lstlisting}
https://github.com/VIB2020/AI1103/blob/main/Assignment%203/code/Assignment%203.py
\end{lstlisting}
%
and latex-tikz codes from 
%
\begin{lstlisting}
https://github.com/VIB2020/AI1103/blob/main/Assignment%203/Assignment%203.pdf
\end{lstlisting}
\section{\Large Problem\\ \large GATE 2015 (EE paper 01 new 2), Q. 27 (Electrical Engg. section)}

Two players A, and B alternately keep rolling a fair dice. The person to get a six first wins the game. Given that player A starts the game, the probability that A wins the game is:\\[5pt]
A: \dfrac{5}{11}\hspace{1cm}
B: \dfrac{1}{2}\hspace{1cm}
C: \dfrac{7}{13}\hspace{1cm}
D: \dfrac{6}{11}
\section{\Large Solution}
Let the random variables X denote the win of A\\
Give that the die is fair\\
$\implies$The probability of getting 6 = $\dfrac{1}{6}$ = p (say)\\[5pt]
$\implies$The probability of getting 6 = $\dfrac{5}{6}$ = q (say)\\[5pt]
P(A wins on the first throw) = p\\[5pt]
\large Constraint:\normalsize A wins the game\\[5pt]
Thus, if A does not get 6 on the first throw then B also should not get 6 on the second throw.
P (A wins on the third throw) = $q^2 p$\\
\imples P (A wins on the $(2n + 1)^t^h $ throw) =\large $q^2^n p$\\[6pt]
\begin{align}
P(X) = \sum_{n=1}^{+\infty} q^2^n p \\[5pt]
= p \sum_{n=1}^{+\infty} (q^2)^n\\[5pt]
= p \left(\dfrac{1}{1 - q^2}\right) = \dfrac{p}{1 - q^2}
=\dfrac{\dfrac{1}{6}}{1 - \dfrac{25}{36}} = \dfrac{6}{11}
\end{align}
\centering
\Large P(A wins the game) = P(X) = \dfrac{6}{11}\\[6pt]
Option D
\end{document}