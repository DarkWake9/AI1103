\documentclass[journal,12pt,twocolumn]{IEEEtran}

\usepackage{setspace}
\usepackage{gensymb}
\usepackage[cmex10]{amsmath}
\usepackage{tikz}
\usetikzlibrary{automata,positioning}
\usepackage{amsthm}

\usepackage{mathrsfs}
\usepackage{txfonts}
\usepackage{stfloats}
\usepackage{bm}
\usepackage{cite}
\usepackage{cases}
\usepackage{subfig}

\usepackage{longtable}
\usepackage{multirow}

\usepackage{enumitem}
\usepackage{mathtools}
\usepackage{steinmetz}

\usepackage{circuitikz}
\usepackage{verbatim}
\usepackage{tfrupee}
\usepackage[breaklinks=true]{hyperref}
\usepackage{graphicx}
\usepackage{tkz-euclide}

\usetikzlibrary{calc,math}
\usepackage{listings}
    \usepackage{color}                                            %%
    \usepackage{array}                                            %%
    \usepackage{longtable}                                        %%
    \usepackage{calc}                                             %%
    \usepackage{multirow}                                         %%
    \usepackage{hhline}                                           %%
    \usepackage{ifthen}                                           %%
    \usepackage{lscape}     
\usepackage{multicol}
\usepackage{chngcntr}

\DeclareMathOperator*{\Res}{Res}

\renewcommand\thesection{\arabic{section}}
\renewcommand\thesubsection{\thesection.\arabic{subsection}}
\renewcommand\thesubsubsection{\thesubsection.\arabic{subsubsection}}

\renewcommand\thesectiondis{\arabic{section}}
\renewcommand\thesubsectiondis{\thesectiondis.\arabic{subsection}}
\renewcommand\thesubsubsectiondis{\thesubsectiondis.\arabic{subsubsection}}

\newcommand{\BEQA}{\begin{eqnarray}}
\newcommand{\EEQA}{\end{eqnarray}}
\newcommand{\define}{\stackrel{\triangle}{=}}
\bibliographystyle{IEEEtran}
\raggedbottom
\setlength{\parindent}{0pt}
\providecommand{\mbf}{\mathbf}
\providecommand{\pr}[1]{\ensuremath{\Pr\left(#1\right)}}
\providecommand{\qfunc}[1]{\ensuremath{Q\left(#1\right)}}
\providecommand{\sbrak}[1]{\ensuremath{{}\left[#1\right]}}
\providecommand{\lsbrak}[1]{\ensuremath{{}\left[#1\right.}}
\providecommand{\rsbrak}[1]{\ensuremath{{}\left.#1\right]}}
\providecommand{\brak}[1]{\ensuremath{\left(#1\right)}}
\providecommand{\lbrak}[1]{\ensuremath{\left(#1\right.}}
\providecommand{\rbrak}[1]{\ensuremath{\left.#1\right)}}
\providecommand{\cbrak}[1]{\ensuremath{\left\{#1\right\}}}
\providecommand{\lcbrak}[1]{\ensuremath{\left\{#1\right.}}
\providecommand{\rcbrak}[1]{\ensuremath{\left.#1\right\}}}
\theoremstyle{remark}
\newtheorem{rem}{Remark}
\newcommand{\sgn}{\mathop{\mathrm{sgn}}}
\providecommand{\abs}[1]{\vert#1\vert}
\providecommand{\res}[1]{\Res\displaylimits_{#1}} 
\providecommand{\norm}[1]{\lVert#1\rVert}
%\providecommand{\norm}[1]{\lVert#1\rVert}
\providecommand{\mtx}[1]{\mathbf{#1}}
\providecommand{\mean}[1]{E[ #1 ]}
\providecommand{\fourier}{\overset{\mathcal{F}}{ \rightleftharpoons}}
%\providecommand{\hilbert}{\overset{\mathcal{H}}{ \rightleftharpoons}}
\providecommand{\system}{\overset{\mathcal{H}}{ \longleftrightarrow}}
	%\newcommand{\solution}[2]{\textbf{Solution:}{#1}}
\newcommand{\solution}{\noindent \textbf{Solution: }}
\newcommand{\cosec}{\,\text{cosec}\,}
\providecommand{\dec}[2]{\ensuremath{\overset{#1}{\underset{#2}{\gtrless}}}}
\newcommand{\myvec}[1]{\ensuremath{\begin{pmatrix}#1\end{pmatrix}}}
\newcommand{\mydet}[1]{\ensuremath{\begin{vmatrix}#1\end{vmatrix}}}
\numberwithin{equation}{subsection}
\makeatletter
\@addtoreset{figure}{problem}
\makeatother
\let\StandardTheFigure\thefigure
\let\vec\mathbf
\renewcommand{\thefigure}{\theproblem}
\def\putbox#1#2#3{\makebox[0in][l]{\makebox[#1][l]{}\raisebox{\baselineskip}[0in][0in]{\raisebox{#2}[0in][0in]{#3}}}}
     \def\rightbox#1{\makebox[0in][r]{#1}}
     \def\centbox#1{\makebox[0in]{#1}}
     \def\topbox#1{\raisebox{-\baselineskip}[0in][0in]{#1}}
     \def\midbox#1{\raisebox{-0.5\baselineskip}[0in][0in]{#1}}

\lstset{
%language=C,
frame=single, 
breaklines=true,
columns=fullflexible
}
\begin{document}
\title{Assignment 3}
\author{Vibhavasu Pasumarti - EP20BTECH11015}
\maketitle
\newpage
\bigskip
\renewcommand{\thefigure}{\theenumi}
\renewcommand{\thetable}{\theenumi}
Download all python codes from 
\begin{lstlisting}
https://github.com/VIB2020/AI1103/blob/main/Assignment%203/code/Assignment%203.py
\end{lstlisting}
%
and latex-tikz codes from 
%
\begin{lstlisting}
https://github.com/VIB2020/AI1103/blob/main/Assignment%203/Assignment%203.pdf
\end{lstlisting}
\section{\Large Problem \\ \large GATE 2015 (EE paper 01 new 2), Q 27 (Electrical Engg section)}
Two players A, and B alternately keep rolling a fair dice. The person to get a six first wins the game. Given that player A starts the game, the probability that A wins the game is:\\[5pt]
    A $\dfrac{5}{11}$ \hspace{1cm}
    B $\dfrac{1}{2}$ \hspace{1cm}
    C $\dfrac{7}{13}$ \hspace{1cm}
    D $\dfrac{6}{11}$ \hspace{1cm}
\section{\Large Solution}
\begin{description}
    \item Let the random variable X denote the win of A.
    \item Given the die is fair.
    \item The probability of getting 6 = $\dfrac{1}{6}$ = p (say)
    \item The probability of NOT getting 6 = $\dfrac{5}{6}$ = q (say)
\end{description}
\begin{center}
Markov chain for the given problem:
\begin{tikzpicture}[font=\sffamily]
\tikzset{node style/.style={state, 
                                    minimum width=1.4cm,
                                    line width=0.5mm,
                                    fill=gray!20!white}}

    \node [node style] at (0, 0)     (a)     {A};
    \node [node style] at (5, 0)     (b)     {B};
    \node [node style] at (2.5, -4.3301) (w) {Winner};
    \draw[every loop,
          auto=right,
          line width=1mm,
          >=latex,
          draw=orange,
          fill=orange]
        
        (a) edge[bend right=20] node {$q = \frac{5}{6}$} (b)
        (b) edge[bend right=20] node {$q = \frac{5}{6}$} (a)
        (a) edge[bend right=10] node {$p = \frac{1}{6}$} (w)
        (b) edge[bend left=10, auto=left] node {$p = \frac{1}{6}$} (w)
\end{tikzpicture}
\end{center}
Transition/Stochastic matrix:\\
\begin{math}
    P = \myvec{
    0 & q & p\\
    q & 0 & p\\
    0 & 0 & 0}
    \implies \lim_{n \to \infty} P^n \xrightarrow{} 0 \small \because p , q < 1\\

    \implies P \text{ is reducible}
    \implies \lim_{n \to \infty}Pr_{AW}(n) = 0\\
\end{math}\\
P$^n$ represents n rolls of dice\\
Given: A starts the game\\
Constraint: A wins the game.\\
The initial operator representing this constraint:\\
A$_1$ = \myvec{p & 0 & 0}\\
First element of each A$_i$ represents the probability of A to win in i$^{th}$ roll\\
\begin{math}
    Pr_{AW}(1) = p\\
    A_2 = A_1 P = \myvec{p & 0 & 0}\myvec{0 & q & 0\\ q & 0 & 0\\ p & p & 0}  = \myvec{0 & qp & 0}\\
    \implies  Pr_{AW}(2) = 0\\
    
    A_3 = A_2 P = \myvec{0 & qp & 0} \myvec{0 & q & 0\\ q & 0 & 0\\ p & p & 0}= \myvec{q^2 p & 0 & 0}\\
    \implies  Pr_{AW}(3) = q^2 p\\
    
    A_4 = \myvec{0 & q^3 p & 0} \implies  Pr_{AW}(4) = 0\\
    A_5 = \myvec{q^4 p & 0 & 0} \implies  Pr_{AW}(5) = q^4 p\\
    \text{By induction}:\\
    A_{2n} = \myvec{0 & q^{2n - 1} p & 0} \implies Pr_{AW}(2n) = 0\\
    A_{2n + 1} = \myvec{q^{2n} & 0 & 0}  \implies Pr_{AW}(2n + 1) = q^{2n} p\\
    \pr{\text{A wins}} = \sum_{n=1}^{\infty} Pr_{AW}(n)\\
    = Pr_{AW}(1) + Pr_{AW}(2) + Pr_{AW}(3) + ....\\
    = p + 0 + q^2p + 0 + q^4p + ....\\
    = \sum_{k=1}^{\infty} q^2^k p
    = p \sum_{k=1}^{\infty} (q^2)^k \hspace{0.5cm} \text{But $|q| <$ 1}\\
    = p \left(\dfrac{1}{1 - q^2}\right)
    =\dfrac{\frac{1}{6}}{1 - \frac{25}{36}} = \dfrac{6}{11}
\end{math}
\centering
\Large \pr{\text{A wins the game}} = \dfrac{6}{11}
\\Option D
\end{document}